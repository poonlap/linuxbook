\def\thairmdefault{kinnari}
\def\thaisfdefault{garuda}
\def\ttdefault{cmtt}

\newcommand{\cmd}[1]{{\latintext\texttt {#1}}}
\newcommand{\mymarginpar}[1]{\marginpar{\raggedright \scriptsize #1}}
%\newcommand{\myvocab}[2]{\nomenclature{#1}{\mbox{}\\ #2\label{vocab:#1}}\mymarginpar{#1 $\blacktriangleright$ หน้า \pageref{vocab:#1}}}
%\newcommand{\myvocab}[3]{\nomenclature[#1]{{\latintext #2}}{\mbox{}\\ #3\label{vocab:#2}}\mymarginpar{#2 $\blacktriangleright$ คำศัพท์หน้า \pageref{vocab:#2}}}
\newcommand{\myvocab}[3]{\nomenclature[#1]{{\latintext #2}}{\mbox{}\\ #3\label{vocab:#2}}\mymarginpar{#2 $\blacktriangleright$\\
#3}}
%\newcommand{\mymemo}[1]{\mymarginpar{\includegraphics[scale=0.4]{icon_memo.eps}\par#1}}
\newcommand{\mymemo}[1]{\mymarginpar{{\large\ding{45}} \hrulefill\par #1}}
%\newcommand{\myexplanation}[2]{\mymarginpar{\includegraphics[scale=0.4]{gnome-terminal.eps} \cmd{#1}\par#2}}
%\newcommand{\myexplanation}[2]{\mymarginpar{\ding{111} \cmd{#1} \ding{111}\par#2}}
\newcommand{\myexplanation}[2]{\mymarginpar{\raisebox{1pt}{\scalebox{.5}{\Ovalbox{\cmd{\myenter}}}} \cmd{#1}\par#2}}
\newcommand{\myxapp}[2]{\mymarginpar{\includegraphics[scale=0.4]{metacity-properties.eps} \cmd{#1}\par#2}}

% practice
\newcommand{\mypractice}[1]{\addtocounter{myprac}{1}\mymarginpar{\textleaf{} \emph{แบบฝึกหัด} \themyprac\ \hrulefill\par #1}}
\newcommand{\myprac}{\item[\addtocounter{myprac}{1}คำถามที่ \themyprac{}.] }

% remember chapter title
\renewcommand{\chaptermark}[1]%
        {\markboth{#1}{}}
% section number and title
\renewcommand{\sectionmark}[1]%
        {\markright{\thesection\ #1}}
\lhead[\fancyplain{}{\bfseries\thepage}]%
        {\fancyplain{}{\bfseries\rightmark}}
\rhead[\fancyplain{}{\bfseries\leftmark}]%
        {\fancyplain{}{\bfseries\thepage}}
\cfoot{}
\rfoot[]{}
\lfoot[]{}


\newcommand{\glossmargin}[2]{\index{#1}\mymarginpar{{\fontencoding{OT1}\fontfamily{cmss}\fontseries{bx}\selectfont #1}:\\ #2 }}

%\makeatletter
%\def\@makechapterhead#1{%
%  \vspace*{50\p@}%
%  {\parindent \z@ \raggedright \normalfont
%    \ifnum \c@secnumdepth >\m@ne
%      \if@mainmatter
%        \hspace*{\stretch{1}} \huge\sffamily\bfseries \@chapapp\space \Huge \thechapter
%        \par\nobreak
%        \vskip 20\p@
%      \fi
%    \fi
%    \interlinepenalty\@M
%    \hspace*{\stretch{1}} \bfseries{\fontfamily{garuda}\fontsize{36pt}{48pt}\selectfont #1}\par\nobreak
%    \vskip 40\p@
%  }}
%\makeatother




\newcommand{\commandline}[1]{{\latintext\tt #1}}
\newcommand{\commandinput}[1]{{\textcolor{blue}{#1}}}

\newcommand{\myleftarrow}{\Pisymbol{psy}{172}}
\newcommand{\myenter}{\Pisymbol{psy}{191}}
\newcommand{\cin}[1]{{\latintext\tt \textcolor{blue}{#1}}\myenter}


\DefineVerbatimEnvironment%
 {MyVerbatim}{Verbatim}
 {formatcom=\latintext\tt,
  frame=single,
  fontsize=\scriptsize,
  framesep=3mm,
%  samepage=true,
%  showspaces=true,
%  showtabs=true,
  commandchars=\\\{\}}



\DefineVerbatimEnvironment%
 {MyMan}{Verbatim}
 {formatcom=\latintext\tt,
  frame=none,
  fontsize=\scriptsize,
  framesep=10mm,
  samepage=true,
%  showspaces=true,
%  showtabs=true,
  commandchars=\\\{\}}

\DefineVerbatimEnvironment%
 {MyEx}{Verbatim}
 {formatcom=\latintext\tt,
%  frame=single,
  fontsize=\scriptsize,
%  framesep=3mm,
%  samepage=true,
%  showspaces=true,
%  showtabs=true,
  commandchars=\\\{\}}

\DefineVerbatimEnvironment%
 {MyCode}{Verbatim}
 {formatcom=\latintext\tt,
  frame=leftline,
  fontsize=\scriptsize,
  framesep=10mm,
%  samepage=true,
%  showspaces=true,
%  showtabs=true,
  commandchars=\\\{\}}

\newcounter{myprac}[chapter]
\renewcommand{\themyprac}{\arabic{chapter}.\arabic{myprac}}
\newcounter{myexample}[chapter]
\renewcommand{\themyexample}{\arabic{chapter}.\arabic{myexample}}


%\newenvironment{myexample}{\begin{MyVerbatim}}
%{\end{MyVerbatim}}

%\def\cursorprompt{\ding{110}}
\def\cursorprompt{\rule{4pt}{6pt}}


\def\mypicture#1#2#3#4{% scale, figure, caption, name
%
\ifthenelse{\isodd{\pageref{fig:#4}}}%
{\parbox{\headwidth}{\center\scalebox{#1}{#2}\caption{#3}\label{fig:#4}}}%
{\leftskip=\moveback\parbox{\headwidth}{\center\scalebox{#1}{#2}\caption{#3}\label{fig:#4}}}}

%%%% rotate
\def\plfigurelongodd#1#2#3#4{% scale, figure, caption, name
%
\ifthenelse{\isodd{\pageref{fig:#4}}}%
{\parbox{\headwidth}{\center\scalebox{#1}{\rotatebox{90}{\includegraphics{#2}}}\caption{#3}\label{fig:#4}}}%
{\leftskip=\moveback\parbox{\headwidth}{\center\scalebox{#1}{\rotatebox{90}{\includegraphics{#2}}}\caption{#3}\label{fig:#4}}}}
\def\plfigurelongeven#1#2#3#4{% scale, figure, caption, name
%
\ifthenelse{\isodd{\pageref{fig:#4}}}%
{\parbox{\headwidth}{\center\scalebox{#1}{\rotatebox{-90}{\includegraphics{#2}}}\caption{#3}\label{fig:#4}}}%
{\leftskip=\moveback\parbox{\headwidth}{\center\scalebox{#1}{\rotatebox{-90}{\includegraphics{#2}}}\caption{#3}\label{fig:#4}}}}
%%%% rotate
\def\plfigure#1#2#3#4{% scale, figure, caption, name
%
\ifthenelse{\isodd{\pageref{fig:#4}}}%
{\parbox{\headwidth}{\center\scalebox{#1}{\includegraphics{#2}}\caption{#3}\label{fig:#4}}}%
{\leftskip=\moveback\parbox{\headwidth}{\center\scalebox{#1}{\includegraphics{#2}}\caption{#3}\label{fig:#4}}}}
\def\plboxfigure#1#2#3#4{% scale, figure, caption, name
\label{fig:#4}\ifthenelse{\isodd{\pageref{fig:#4}}}{\fbox{\parbox{\headwidth}{\center\scalebox{#1}{\includegraphics{#2}}\caption{#3}}}}{\leftskip=\moveback\label{fig:#4}\fbox{\parbox{\headwidth}{\center\scalebox{#1}{\includegraphics{#2}}\caption{#3}}}}}
\def\plfiguretwo#1#2#3#4#5{% scale, figure, caption, name
%
\ifthenelse{\isodd{\pageref{fig:#5}}}%
{\parbox{\headwidth}{\center\scalebox{#1}{\includegraphics{#2}~~~\includegraphics{#3}}\caption{#4}\label{fig:#5}}}%
{\leftskip=\moveback\parbox{\headwidth}{\center\scalebox{#1}{\includegraphics{#2}~~~\includegraphics{#3}}\caption{#4}\label{fig:#5}}}}
\def\plfigurethree#1#2#3#4#5#6{% scale, figure, caption, name
%
\ifthenelse{\isodd{\pageref{fig:#6}}}%
{\parbox{\headwidth}{\center\scalebox{#1}{\includegraphics{#2}~~~\includegraphics{#3}~~~\includegraphics{#4}}\caption{#5}\label{fig:#6}}}%
{\leftskip=\moveback\parbox{\headwidth}{\center\scalebox{#1}{\includegraphics{#2}~~~\includegraphics{#3}~~~\includegraphics{#4}}\caption{#5}\label{fig:#6}}}}
\def\arrowdown{\Pisymbol{psy}{175}}
\def\arrowleft{\Pisymbol{psy}{172}}
\def\arrowup{\Pisymbol{psy}{173}}
\def\arrowright{\Pisymbol{psy}{174}}
\def\kk#1{\Ovalbox{\cmd{#1}}}
%\def\mycomment#1{\hspace*{\stretch{1}}\shabox{\myleftarrow{} \thtt{#1}}}
\def\mycomment#1{\hspace*{\stretch{1}}\myleftarrow{} \thtt{#1}}


\usepackage{sverb}

\newcounter{filenum}
\makeatletter
\newenvironment{thwbr}%
  {\verbwrite{\jobname.\thefilenum.th}}
  {\endverbwrite}
\def\wbrin{\@input@{\jobname.\thefilenum.wbr}\stepcounter{filenum}}

\def\thraw#1{%
\newwrite\outputfile
\immediate\openout\outputfile=\jobname.\thefilenum.th\relax% th: Thai Text
\write\outputfile{\protect{}{\thaitext{}#1}}
\closeout\outputfile
}


\setlength{\columnsep}{3em}
\def\printindex#1#2{

\begin{multicols}{2}[\chapter*{\makebox[\headwidth][r]{\sffamily\bfseries #2}}]

\markboth{#2}{#2}
  \@input{#1.ind}\end{multicols}}

%%\def\@@@nomenclature[#1]#2#3{% 
%%\def\@tempa{#2}\def\@tempb{#3}% 
%%\protected@write\@glossaryfile{}% 
%%{\string\glossaryentry{#1\nom@verb\@tempa @{\nom@verb\@tempa}&% 
%%\begingroup\nom@verb\@tempb\protect\nomeqref{\theequation}% 
%%|nompageref}{\thepage}}% 
%%\endgroup \@esphack}

%\def\@@@nomenclature[#1]#2#3{% 
%\def\@tempa{#2}\def\@tempb{#3}% 
%\protected@write\@glossaryfile{}% 
%{\string\glossaryentry{\item[#1]\mbox{}\\
%#2}}
%\endgroup}


%%\def\theglossary{% 
%%\@ifundefined{chapter}{\section*}{\chapter*}{\nomname}% 
%%%\nompreamble \begin{longtable}[l]{@{}ll@{}}} \def\endtheglossary{% 
%%\nompreamble \begin{longtable}[l]{l|p{.8\textwidth}}} \def\endtheglossary{% 
%%\end{longtable}% 
%%\nompostamble}

\def\theglossary{% 
\@ifundefined{chapter}{\section*}{\chapter*}{\nomname}% 
%\nompreamble \begin{longtable}[l]{@{}ll@{}}} \def\endtheglossary{% 
\markboth{รวมคำศัพท์คอมพิวเตอร์}{พูลลาภ วีระธนาบุตร \thtt{http://linux.thai.net/Members/poonlap}}\nompreamble \begin{description}} \def\endtheglossary{% 
\end{description}% 
\nompostamble}

%%
%% Modified \chapter
%%
\def\mhrulefill{\leavevmode\leaders\hrule height 2pt \hfill\kern\z@}
\long\def\@makechapterhead#1{%
  \vspace*{50\p@}%
  {\parindent \z@ \raggedright \normalfont
    \ifnum \c@secnumdepth >\m@ne
      \if@mainmatter
        \makebox[\headwidth][r]{\mhrulefill\hspace{15pt}\fontsize{29.85}{32}%
	\usefont{LTH}{garuda}{b}{n} \@chapapp\space %
	\fontsize{61.9}{65}\usefont{OT1}{cmss}{bx}{n}\thechapter}
%	\huge\bfseries \@chapapp\space \thechapter
        \par\nobreak
        \vskip 20\p@
      \fi
    \fi
    \interlinepenalty\@M
    \makebox[\headwidth][r]{\fontsize{35.83}{40}%
	\usefont{LTH}{garuda}{b}{n} #1\par\nobreak}
%    \Huge \bfseries #1\par\nobreak
    \vskip 40\p@
  }}
\renewcommand\chapter{\if@openright\cleardoublepage\else\clearpage\fi
                    \thispagestyle{empty}%
                    \global\@topnum\z@
                    \@afterindentfalse
                    \secdef\@chapter\@schapter}
\def\@chapter[#1]#2{\ifnum \c@secnumdepth >\m@ne
                       \if@mainmatter
                         \refstepcounter{chapter}%
                         \typeout{\@chapapp\space\thechapter.}%
                         \addcontentsline{toc}{chapter}%
                                   {\protect\numberline{\thechapter}#1}%
                       \else
                         \addcontentsline{toc}{chapter}{#1}%
                       \fi
                    \else
                      \addcontentsline{toc}{chapter}{#1}%
                    \fi
                    \chaptermark{#1}%
                    \addtocontents{lof}{\protect\addvspace{10\p@}}%
                    \addtocontents{lot}{\protect\addvspace{10\p@}}%
                    \if@twocolumn
                      \@topnewpage[\@makechapterhead{#2}]%
                    \else
                      \@makechapterhead{#2}%
                      \@afterheading
                    \fi}
%%
%% Modified \chapter* 
%%
\def\@makeschapterhead#1{%
  \vspace*{50\p@}%
  {\parindent \z@ \raggedright
    \normalfont
    \interlinepenalty\@M
    \makebox[\headwidth][r]{\fontsize{35.83}{40}%
	\usefont{LTH}{garuda}{b}{n} #1\par\nobreak}
%    \Huge \bfseries  #1\par\nobreak
    \vskip 40\p@
  }}
%%
%% Leave blank page
%%
\def\cleardoublepage{\clearpage\if@twoside \ifodd\c@page\else
  \hbox{}
  \vspace*{\fill}
  \thispagestyle{empty}
  \newpage
  \if@twocolumn\hbox{}\newpage\fi\fi\fi}
\makeatother


\makeatletter
%\def\nomgroup#1{\ifodd\c@page\marginpar{\break\break\hfill\textinit{#1}\par}\else\marginpar{\textinit{#1}\par}\fi}
%\def\nomgroup#1{\ifodd\c@page\marginpar{\mbox{}\par\mbox{}\par\mhrulefill\ {\fontsize{24}{26}\usefont{OT1}{cmss}{bx}{n} #1}\par}\else\marginpar{\mbox{}\par\mbox{}\par{\fontsize{24}{26}\usefont{OT1}{cmss}{bx}{n} #1 \mhrulefill}\par}\fi}
\def\nomgroup#1{\marginpar{\mbox{}\par\mbox{}\par\mhrulefill\ {\fontsize{24}{26}\usefont{OT1}{cmss}{bx}{n} #1 \mhrulefill}\par}}
\def\bookimg#1#2{\ifodd\c@page\marginpar{\hfill\includegraphics[scale=#1]{#2}}\else\marginpar{\includegraphics[scale=#1]{#2}}\fi}


% Define thai charactor translation
% thaiAlph is full alphabet set
% thaialph is reduced alphabet set
\def\thaialph#1{\expandafter\@thaialph\csname c@#1\endcsname}
\def\thaiAlph#1{\expandafter\@thaiAlph\csname c@#1\endcsname}
\def\@thaialph#1{%
  \ifcase#1\or ก\or ข\or ค\or ง\or จ\or ฉ\or ช\or ซ\or ฌ\or ญ\or ฎ\or
   ฏ\or ฐ\or ฑ\or ฒ\or ณ\or ด\or ต\or ถ\or ท\or ธ\or น\or บ\or ป\or ผ\or 
   ฝ\or พ\or ฟ\or ภ\or ม\or ย\or ร\or ล\or ว\or ศ\or ษ\or ส\or ห\or ฬ\or อ\or
     ฮelse\@ctrerr\fi}
\def\@thaiAlph#1{%
  \ifcase#1\or ก\or ข\or ฃ\or ค\or ฅ\or ฆ\or ง\or จ\or ฉ\or ช\or ซ\or
   ฌ\or ญ\or ฎ\or ฏ\or ฐ\or ฑ\or ฒ\or ณ\or ด\or ต\or ถ\or ท\or ธ\or น\or
    บ\or ป\or ผ\or ฝ\or พ\or ฟ\or ภ\or ม\or ย\or ร\or ฤ\or ล\or ฦ\or ว\or
     ศ\or ษ\or ส\or ห\or ฬ\or อ\or ฮelse\@ctrerr\fi}
\renewcommand\appendix{\par
  \setcounter{chapter}{0}%
  \setcounter{section}{0}%
  \gdef\@chapapp{\appendixname}%
  \gdef\thechapter{{\thaitext\@thaialph\c@chapter}}}

\makeatother

\def\thtt#1{{\fontencoding{LTH}\fontfamily{pmono}\selectfont #1}}

\def\cindex#1{\index{command}{#1@{\latintext\texttt{#1}}}}
\def\findex#1{\index{command}{#1@{\latintext\texttt{#1}} (ไฟล์)}}
\def\dindex#1{\index{command}{#1@{\latintext\texttt{#1}} (ไดเรกทอรี)}}
\def\gindex#1{\index{general}{#1}}
%\def\findex#1{\index{filei}{#1}}


%
% Example related
%
%\definecolor{Mygrey}{rgb}{0.8,0.8,0.8}
\definecolor{shadecolor}{rgb}{0.8,0.8,0.8}
\shadecolor{lightgray}
%\theoremseparator{\hrulefill}
\theoremstyle{break}
%\theorembodyfont{\latintext\tt}
\theoremheaderfont{\scriptsize}
\theorempreskipamount=18pt
%\theorempreskipamount=0pt
\theorempostskipamount=18pt
%\theorempostskipamount=-16pt
\theoremindent=12pt
%\newshadedtheorem{MyExample}{{\thaitext ตัวอย่างที่}}[chapter]
\newtheorem{MyExample}{{\thaitext ตัวอย่างที่}}[chapter]
%\newframedtheorem{MyExample}{ตัวอย่างที่}[chapter]


\newcommand\bs{\char92}
\newcommand\qq{\char32}
\newcommand\dollar{\char36}
\newcommand\dq{"}
\newcommand\cmdit[1]{\cmd{\textit{#1}}}
\newcommand\wrap{\ \rotatebox{-90}{$\curvearrowright$}}

\newcommand\cmdentry[1]{\subsection*{\cmd{#1}}\label{cmd:#1}\cindex{#1}}
\newcommand{\refcmd}[1]{\mymarginpar{\raisebox{1pt}{\scalebox{.5}{\Ovalbox{\cmd{\myenter}}}} \cmd{#1} อ้างอิงหน้า \pageref{cmd:#1}}}

%%%%%%%%% can be used as column
%\begin{figure}[tb]
%\begin{boxedminipage}[!t]{\textwidth}
%
%\end{boxedminipage}
%\end{figure}

\newcommand{\graybox}[1]{\psboxit{box .7 setgray fill}{\addtolength{\fboxsep}{-5pt}\spbox{#1}\addtolength{\fboxsep}{5pt}}}


% key pad figure
\newcommand{\keydel}{\raisebox{-4pt}{\scalebox{.25}{\includegraphics{key_del.eps}}}}
\newcommand{\keyctrl}{\raisebox{-4pt}{\scalebox{.25}{\includegraphics{key_ctrl.eps}}}}
\newcommand{\keyc}{\raisebox{-4pt}{\scalebox{.25}{\includegraphics{key_c.eps}}}}
\newcommand{\abb}{--- \thtt{แสดงผลต่อไปเรื่อยๆ} ---}
